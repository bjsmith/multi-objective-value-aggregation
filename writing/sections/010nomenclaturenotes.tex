\section*{nomenclature notes}
nomenclature we need to get straight. Robert and Roland, these can definitely be changed if you prefer other nomenclature. this set of notes is just intended for us to syncrohonize and ensure we all use the same language consistently throughout the paper:

\begin{itemize}
    \item names of each of the agents
        \begin{itemize}
            \item SFELLA not SFLLA or SFLLA2. Can use a subscript for SFLLA2 like $SFLLA_{\text{rt}}$ to indicate reward transformation, but it's discoraged. Discussion of the reward transformation is not usually extensive enough to use a subscript. 
            \item same for other agents; reward transformation versions should be denoted with a $x_{\text{rt}}$ if it isn't clear from context
            \item tables and graphs should all conform.
            \item TLO$^\text{A}$ not $TLO_A$. using a superscript character is an acceptable substitute in graphs.
            \item Use `LinearSum' not LIN\_SUM or other form.
        \end{itemize}
    \item names of each of the environments follows Vamplew's use: CamelCase and abbreviations. do not use a space for BreakableBottles or UnbreakableBottles. Can abbreviate as BB and UB. Environment names are always capitalized. Can use ``Bottles environments'' to refer to BB and UB collectively.
    \item Objectives: We have `primary' ($R^P$) and `alignment' ($R^A$) objectives, as well as the $R^*$ `performance metric', not `performance objective', to keep it clearly delineated. `Performance objective' should NOT be used to refer to the primary objective. `Evaluation objective' is deprecated; use `performance metric'. `Primary' and `alignment' are not specially capitalized.
    \item you don't have `performance' on \RA{} and \RP{}; you have `scores' if necessary, but `performance' is reserved for \RStar{}.
    \item `Primary re-scaling', not `primary scale perturbation' or `reward scale perturbation'. `perturbation' is deprecated in favor of `re-scaling'
    \item `scaling' refers to changing the size of the rewards given by the environment as we have done in Experiment 1. it's linear. It should be thought of as a change in the environment itself.
    \item `transformation' generally refers to the non-linear transforms we apply to each objective. often [in case of q-value] they are combined following transform, though you could imagine method where they are not.
    \item experiment 1 is `Q-value transform', try to avoid `utility transform' although it's an OK synonym. Experiment 2 is `reward transform'. don't mix these up.
    \begin{itemize}
        \item Reward - the measure code applies at each time step
        \item Q value - the predicted reward (in state * action)
        \item Utility - the inspiration for our transformations from economics
    \end{itemize}
    \item Roland: search reward, reinforcement, utility, make sure words are used appropriately.
    \item Applying granularity is not `scaling' at all, although it could in principle be presented over different levels of scaling, I don't think that is what we've done in our paper
    \item Do we need to standardize terms discussing granularity?
    \item In our paper, `Online' performance is performance during learning; `offline' performance is performance after learning (this may not be generally true). Don't capitalize the terms online and offline.
    
\end{itemize}

\section{Future directions}

\begin{itemize}
    \item Maybe in the future do a statistical comparison between reward transformation and Q value transformation (varying the location of the transformation between same function and scale or granularity)
\end{itemize}
 
\section*{additional editing tasks}
\begin{itemize}
    \item @Roland: please revisit Results sections and hopefully discussion sections and confirm you're happy with them
    \item @Roland: For Exp3, which sokoban should we show? Plots just choose one, tables can have multiple -- On the plots I propose to show Sokoban rew 0.01 because it has best scores and having better scores was the motivation for scaling.
    \item Table 1: which items do we want on here? - probably keep SEBA, even though EEBA is a bit better. This is probably fine as is.
    \item algorithms need to be improved - probably no changes. say what the rounding levels are - add reference to plot near the granularity formula?
    \item Label tables and graphs nicely - Ben to do
    \item refine the exact comparisons shown in tables and graphs. might leave this until we have substantially finished writing. What things are we actually trying to say?
    \begin{itemize}
        \item we need to discuss the graphs together after we've got the main text substantially finished.
        \item Label agents correctly
        \item Granularity plots: leave out scale 100 because changes dwarf other items and make it hard to see. Add something in the caption to say why we did that
        \item should we do middle 80\% instead of middle 90\%?
    \end{itemize}
    \item Make sure the reason for every function being here is clear, including ELA, LELA, SEBA, LinearSum, etc We can remove some if it is more appropriate.
    \begin{itemize}
        \item SEBA: I'm satisfied enough that we have adequately justified SFELLA and SEBA.
        \item take out LELA and ELA maybe. Maybe leave only formulas and refer to how a similar one is used in economy.
        \item Why do we include LinearSum as a second baseline? need to specify
        \item Linear sum - because it is most intuitive naive choice implemented by people using RL with multiple objectives
        \item \tloA{} - because Vamplew used it and because it is an useful construction though complicated to use
    \end{itemize}
    \item @Roland: Outline why we've scaled sokoban for the locations where it has been scaled. - Done
    \item Table 2: what do we want here. do we still want to change it to compare vs. base granularity? - NO, decided to keep it the same.
    \item @roland: Explain why granularity results fall below \tloA{} and how introducing non-arbitrary offset might confirm our granularity hypothesis more fully. - DONE
    \item Introduction, last paragraph, and , are about Goodharting - @roland to edit - edited
    \item 1.3 Pluralistic value systems, last pararaph has been deleted.
        
    \item section `Aggregation functions'. What is the scaling factor? The scaling factor is applied to reinforcement but the transform is applied to Q-values, so it's not really correct to present this as a set of steps. - still don't love this but we have an awkward set of ideas to communicate and this is as best as we can do.-BJS
    \item Experiment 3: describe the granularity transform precisely in terms of how it relates to our earlier functions. - which equation - do you mean it applies before Equation 2?
    \item Roland to propose a definition for the `soft' in `soft maximin'.
    Here we propose a possible definition for the phrase "soft maximin" used in the title. The word "soft" represents consideration of different objectives or application of rules in a flexible manner without sharp boundaries. The word "maximin" originates from a concept by John Rawls. The "min" function is useful for aggregating multiple objectives in such a way that the objective with a lowest score at any particular moment has most influence on the final outcome, hence the "maxi" part. It can be conceptually related to the conjunction operator as well as the concept of vetoing. Now combining the "min" function with the "soft" property means that our proposed functions approximate the "min" function but are smooth, without sharp turns. That makes the derivative of our functions continuous and enables having different feedback to the agent at all possible rewards. This facilitates continuous feedback for even small changes in rewards, therefore improving the learning speed.
    \item Robert to visualise the soft maximin idea
%    \item Ben to change the descriptions within figure 4 and 5, and 6 captions nad x-axes of "Scaling" to "transform" or similar approrpriate, and replace "peerformance" with "primary". Should probably say "Primary/Alignment transform: average online performance"; then x-axis caption should have "Primary/Alignment Transform magnitude" - DONE
    %\item Ben to change Experimetn 3 figure TO DO: sort to more easily show the change over time, smallest granularities at top and then bigger ones at the bottom. - DONE
\end{itemize}
    
 