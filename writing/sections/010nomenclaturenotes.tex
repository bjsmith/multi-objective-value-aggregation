\section*{nomenclature notes}
nomenclature we need to get straight. Robert and Roland, these can definitely be changed if you prefer other nomenclature. this set of notes is just intended for us to syncrohonize and ensure we all use the same language consistently throughout the paper:

\begin{itemize}
    \item names of each of the agents
        \begin{itemize}
            \item SFELLA not SFLLA or SFLLA2. Can use a subscript for SFLLA2 like $SFLLA_{\text{rt}}$ to indicate reward transformation, but it's discoraged. Discussion of the reward transformation is not usually extensive enough to use a subscript. 
            \item same for other agents; reward transformation versions should be denoted with a $x_{\text{rt}}$ if it isn't clear from context
            \item tables and graphs should all conform.
            \item $\text{TLO}^\text{A}$ not $TLO_A$. using a superscript character is an acceptable substitute in graphs.
            \item Use `LinearSum' not LIN\_SUM or other form.
        \end{itemize}
    \item names of each of the environments: CamelCase and abbreviations. do not use a space for BreakableBottles or UnbreakableBottles. Can abbreviate as BB and UB. 
    \item Objectives: We have `primary' ($R^P$) and `alignment' ($R^A$) objectives, as well as the $R^*$ `performance metric', not `performance objective', to keep it clearly delineated. `Performance objective' should NOT be used to refer to the primary objective.
    \item `scaling' refers to changing the size of the rewards given by the environment as we have done in Experiment 1. it's linear. It should be thought of as a change in the environment itself.
    \item `transformation' generally refers to the non-linear transforms we apply to each objective. often [in case of q-value] they are combined following transform, though you could imagine method where they are not. We have `Q-value transform' and `reward transform' but should it be `utility transform'?
    \item Applying granularity is not `scaling' at all, although it could in principle be presented over differnet levels of scaling, I don't think that is what we've done in our paper
    \item `Online' performance is performance during learning; `Offline' performance is performance after learning.
\end{itemize}

 
 
\section*{additional editing tasks}
\begin{itemize}
    \item Table 1: which items do we want on here?
    \item algorithms need to be improved
    \item Label tables and graphs nicely
    \item refine the exact comparisons shown in tables and graphs. might leave this until we have substantially finished writing. What things are we actually trying to say?
    \item Make sure the reason for every function being here is clear, including ELA, LELA, SEBA, LinearSum, etc We can remove some if it is more appropriate.
    \item Outline why we've scaled sokoban for the locations where it has been scaled.
    \item remove inappropriate descriptions of `Scaling' in Exp 3
\end{itemize}
 