
%%let's use a standard format for the discussion
%https://www.ncbi.nlm.nih.gov/pmc/articles/PMC4548568
%1) On what issue we have to concentrate, discuss or elaborate? 2) What solutions can be recommended to solve this problem? 3) What will be the new, different, and innovative issue? 4) How will our study contribute to the solution of this problem An introductory paragraph in this format is helpful to accomodate reader to the rest of the Discussion section. However summarizing the basic findings of the experimental studies in the first paragraph is generally recommended by the editors of the journal.[5]

%In the last paragraph of the Discussion section “strong points” of the study should be mentioned using “constrained”, and “not too strongly assertive” statements. Indicating limitations of the study will reflect objectivity of the authors, and provide answers to the questions which will be directed by the reviewers of the journal. On the other hand in the last paragraph, future directions or potential clinical applications may be emphasized.

Of the five agents tested, one in particular, SFLLA, performed equally or better to the state-of-the-art agent ($\text{TLO}^\text{A}$) when reward scaling was perturbed. In the BreakableBottles task particularly, SFLLA performed better while $\text{TLO}^\text{A}$ declined in performance as the primary utility was magnified. This indicates that the SFLLA function is robust to changes in the incentive structure of the task in ways that the thresholded method $\text{TLO}^\text{A}$ is not. The SFLLA model heavily penalizes any change in $x$ where $x>0$, i.e., for primary utility (Figure~\ref{fig:transform_functions}, Right). This is a middle ground between ELA and LELA, which enables it to be robust but not completely insensitive to large perturbations of primary utility. Compared to the ELA function, the SFLLA maintains more sensitivity to $x$ where $x>0$, whereas for $x$ values significantly above 0, $f_{\text{ELA}}(x)$ becomes almost completely insensitive to $x$. 



\subsection{Future directions}

When applying exponential transforms  on each objective and then combining them in linear fashion, the scale of the operation is quite important. The scales were designed to respond to z-scored input functions, i.e., most values typically appear between -3 an 3 (Figure~\ref{fig:transform_functions}). However, the environments tested here have input functions that vary much more widely.

It may be helpful, for each objective, to scale the distribution of possible rewards to zero-deviation of 1, without centering on the mean. This is a `zero-deviation' rather than a standard deviation, because the mean absolute difference from the mean may not be 1; instead the mean absolute difference from zero is 1 (or -1). A useful extension would be to somehow build a learning function to understand the distribution of possible rewards.




\subsection{Limitations}

Some models of AI alignment focus on \cite{russell2019human} aligning to human preferences within a probabilistic, perhaps a Bayesian uncertainty modeling framework.  In this model, it isn't necessary to explicitly model multiple competing human objectives. Instead, conflict between human values may be learned and represented implicitly as uncertainty over the action humans prefer. Where sufficient uncertainty exists, a sufficiently intelligent agent motivated to align to human preferences might respond by requesting clarification about the correct course of action from a human. This has in common the `clarification request' under `decision paralysis' described in this paper \footnote{I think we do need to add a \textit{brief} note about this somewhere!}. But it remains to be seen whether a preference alignment approach can eliminate the need for explicit modeling of competing values.

We considered ways to implement maximin approaches such as that described by \cite{vamplew_human-aligned_2018}. In a maximin approach, an agent always selects the action with the maximum value where the value of each action is determined by its minimum evaluation across a set of objectives. Although in this paper, we tested agents with incentive structures with only two objectives, there is no reason a hypothetical agent could not have many objectives. With a sufficiently large number of objectives, it may be that in some states, any possible action would evaluate negatively on some objective or another. In that case an agent may reach `decision paralysis', where the agent is unable to proceed. In that instance, an agent might request clarification from a human overseer. This might lead to iterative improvement or tuning of the agent's goals.

\subsection{Conclusion}

Continuous non-linear transformation functions could offer a way to find a compromise between multiple objectives where a specific threshold cannot be identified. This could be useful when the trade-offs between objectives are not absolutely clear. We provide evidence that one such non-linear transformation function, SFLLA, is better able to respond to perturbations in the scale of a reward penalty.