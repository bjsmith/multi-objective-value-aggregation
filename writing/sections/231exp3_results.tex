\subsection{Results}



\begin{table}[t]
\footnotesize
  \caption{Performance over granularity levels relative to \tloA{}. Items are significantly different from \tloA{} when marked *$p<0.05$, ** $p <0.01$, *** $p<0.001$.}
  \label{tab:granularity_significance}
\begin{adjustbox}{width=\columnwidth}

\begin{tabular}{>{\raggedright\arraybackslash}p{5em}>{\raggedleft\arraybackslash}p{4em}>{\raggedright\arraybackslash}p{4.5em}rrrr}
\toprule
Environment & Primary Objective Granularity & Alignment Objective Granularity & LinearSum & SFELLA & EEBA & TLO$^A$\\
\midrule
 &  & 0.01 & 1.47*** & 6.61*** & 1.51** & \\

 &  & 1.00 & 1.57** & 4.08*** & 1.46*** & \\

 & \multirow[t]{-3}{4em}{\raggedleft\arraybackslash 0.00} & 100.00 & 1.44*** & 1.39*** & 1.58* & \\

 & 0.01 &  & 1.38*** & 6.47*** & 1.42*** & \\

 & 1.00 &  & 1.46*** & 6.37*** & 1.09*** & \\

\multirow[t]{-6}{5em}{\raggedright\arraybackslash BB} & 100.00 & \multirow[t]{-3}{4.5em}{\raggedright\arraybackslash 0.00} & 1.49** & -40.38*** & -41.39*** & \multirow[t]{-6}{*}{\raggedleft\arraybackslash 1.82}\\
\cmidrule{1-7}
 &  & 0.01 & -0.48*** & 4.02 & 1.50*** & \\

 &  & 1.00 & -0.51*** & 4.64*** & 1.39*** & \\

 & \multirow[t]{-3}{4em}{\raggedleft\arraybackslash 0.00} & 100.00 & -0.45*** & -1.02*** & -1.08*** & \\

 & 0.01 &  & -0.47*** & 3.96 & 0.92*** & \\

 & 1.00 &  & -0.46*** & 3.80 & 1.17*** & \\

\multirow[t]{-6}{5em}{\raggedright\arraybackslash Doors} & 100.00 & \multirow[t]{-3}{4.5em}{\raggedright\arraybackslash 0.00} & -0.38*** & -39.01*** & -41.87*** & \multirow[t]{-6}{*}{\raggedleft\arraybackslash 3.96}\\
\cmidrule{1-7}
 &  & 0.01 & -457.70*** & -417.15*** & -457.66*** & \\

 &  & 1.00 & -457.60*** & -457.74*** & -457.52*** & \\

 & \multirow[t]{-3}{4em}{\raggedleft\arraybackslash 0.00} & 100.00 & -457.80*** & -457.78*** & -457.77*** & \\

 & 0.01 &  & -457.62*** & -408.62*** & -457.60*** & \\

 & 1.00 &  & -457.53*** & -341.83*** & -456.24*** & \\

\multirow[t]{-6}{5em}{\raggedright\arraybackslash Sokobanpen10.0} & 100.00 & \multirow[t]{-3}{4.5em}{\raggedright\arraybackslash 0.00} & -457.60*** & -206.95*** & -195.94*** & \multirow[t]{-6}{*}{\raggedleft\arraybackslash -153.60}\\
\cmidrule{1-7}
 &  & 0.01 & -18.01* & -18.01 & -17.99 & \\

 &  & 1.00 & -17.98 & -18.60* & -18.59* & \\

 & \multirow[t]{-3}{4em}{\raggedleft\arraybackslash 0.00} & 100.00 & -18.02** & -48.86*** & -48.84*** & \\

 & 0.01 &  & -17.97 & -17.91** & -17.89*** & \\

 & 1.00 &  &  & -20.83*** & -19.23*** & \\

\multirow[t]{-6}{5em}{\raggedright\arraybackslash Sokobanrew0.01} & 100.00 & \multirow[t]{-3}{4.5em}{\raggedright\arraybackslash 0.00} & \multirow[t]{-2}{*}{\raggedleft\arraybackslash -18.01*} & -23.52*** & -23.09*** & \multirow[t]{-6}{*}{\raggedleft\arraybackslash -17.97}\\
\cmidrule{1-7}
 &  & 0.01 & -182.67 & -182.62 & -182.31 & \\

 &  & 1.00 & -182.42 & -188.27 & -189.02* & \\

 & \multirow[t]{-3}{4em}{\raggedleft\arraybackslash 0.00} & 100.00 & -182.85** & -491.56*** & -491.67*** & \\

 & 0.01 &  & -182.59 & -181.69*** & -181.02*** & \\

 & 1.00 &  & -182.44 & -199.79*** & -193.07*** & \\

\multirow[t]{-6}{5em}{\raggedright\arraybackslash Sokobanrew0.01pen10.0} & 100.00 & \multirow[t]{-3}{4.5em}{\raggedright\arraybackslash 0.00} & -182.41 & -242.13*** & -229.34*** & \multirow[t]{-6}{*}{\raggedleft\arraybackslash -182.31}\\
\cmidrule{1-7}
 &  & 0.01 & 28.72*** & 27.94*** & 28.73*** & \\

 &  & 1.00 & 28.71*** & 28.77*** & 28.70*** & \\

 & \multirow[t]{-3}{4em}{\raggedleft\arraybackslash 0.00} & 100.00 & 28.77*** & 28.72*** & 28.76*** & \\

 & 0.01 &  & 28.76*** & 27.91*** & 28.73*** & \\

 & 1.00 &  & 28.73*** & 27.79*** & 28.64*** & \\

\multirow[t]{-6}{5em}{\raggedright\arraybackslash UB} & 100.00 & \multirow[t]{-3}{4.5em}{\raggedright\arraybackslash 0.00} & 28.74*** & -8.23*** & -13.27*** & \multirow[t]{-6}{*}{\raggedleft\arraybackslash 27.10}\\
\bottomrule
\end{tabular}

\end{adjustbox}
\end{table}


\begin{figure}
  %\centering
  %\includegraphics[width=\columnwidth]{output/onlinepen.pdf}
  \includegraphics[width=\columnwidth]{output/multirun_n100_pilot_granularityonline_RewGranularity.pdf}
  \includegraphics[width=\columnwidth]{output/multirun_n100_pilot_granularityonline_PenGranularity.pdf}
  \caption{By creating `granularity' for our non-linear transform agents, we can simulate similarity with \tloA{}, which might be considered as a custom-tuned form of granularity.
  }
   \label{fig:exp3_main}
   \Description{By creating `granularity' for our non-linear transform agents, we can simulate similarity with \tloA{}, which might be considered as a custom-tuned form of granularity.}
 \end{figure}
 
For SFELLA, as expected, performance declined as granularity increased, particularly in the BreakableBottles environment where it previously had a clear advantage over \tloA{} (Figure~\ref{fig:exp3_main}). For UnbreakableBottles, performance declined as Utility Objective Granularity (???) was increased. In the Doors environment, performance declined, from significantly better (??? Table~\ref{tab:granularity_significance}) than \tloA{} to much, much worse. In the Sokoban environment, as in Experiment 1, \tloA{} was the better performer and remained so.

The result confirms that where SFELLA performs well, it does so because it avoids `granularity' and is sensitive to changes in utility right across teh scale. In contrast, \tloA{} is sometimes insensitive to changes that exceed its threshold. Where it is well tuned, it performs well, or even better, than othere algorithms, but when not well-tunred, it performs less well.



